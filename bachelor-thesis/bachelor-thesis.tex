\RequirePackage[l2tabu, orthodox]{nag} % Warn about outdated commands/packages.
% The report class uses some outdated commands, about which nag will complain.
% You can just ignore these warnings.

\documentclass[11pt, a4paper]{report} % Sets font and paper size.

%%% General formatting packages (order is important, so don't sort) %%%
\usepackage{amsmath} % More equation formatting.
\usepackage[dutch, english]{babel} % Language specific quirks.
\usepackage{booktabs} % Improved tables.
\usepackage[font=small]{caption} % Better caption formatting.
\usepackage{fancyhdr} % Modification of headers and footers.
\usepackage[T1]{fontenc} % Makes one unicode character of special input (e.g. ö).
\usepackage[margin=1.25in]{geometry} % Control page layout.
\usepackage{float} % More control over image positions.
\usepackage{graphicx} % Include graphics. Use '\graphicspath' to locate files in a different folder.
\usepackage[utf8]{inputenc} % Special characters (e.g. trema) can be entered directly: .tex file has to be saved using UTF-8 encoding.
\usepackage{lmodern} % Alternative font because 'fontenc' package does not work with default.
\usepackage{microtype} % Improves character spacing.
\usepackage{natbib} % Provides (author, year) references. \citet: textual, \citep: parenthetical
\usepackage{physics} % Provides physics macros such as Dirac notation.
\usepackage{tikz} % Draw diagrams and figures.
\usepackage{url} % Allow inclusion of urls in text.
\usepackage{siunitx} % SI unit formatting and scientific notation.
\usepackage{subcaption} % Allow subcaptions.
\usepackage[nottoc]{tocbibind} % Include references in table of contents.
\usepackage[colorinlistoftodos]{todonotes} % Add todo notes.
\usepackage{varioref} % Automatically put reference to page in reference. Use \vref.
\usepackage{cleveref} % Automate "equation (...)" reference. Use \vref.

%%% Additional options %%%
\pagestyle{fancy} % Set header style.
\setlength{\headheight}{14pt}

%%% Personal Information %%%
\newcommand\TITLE{Monte Carlo Simulations of the 3-State Potts Model in 2D}
\newcommand\THESISFORM{Bachelor Project Physics and Astronomy, size 15 EC\\conducted between 29-03-2016 and xx-xx-2016}
\newcommand\INSTITUTE{Instituut voor Theoretische Fysica Amsterdam}
\newcommand\FACULTY{Faculteit der Natuurwetenschappen, Wiskunde en Informatice}
\newcommand\UNIVERSITY{Universiteit van Amsterdam}
\newcommand\AUTHOR{Teun Zwart (10499873)}
\newcommand\SUPERVISOR{dr. Phillipe Corboz}
\newcommand\SECONDASSESSOR{dr. Edan Lerner}
\newcommand\UNIVERSITYLOGO{UvA-logo.png} % Uncomment line below and add name of logo file.

\graphicspath{{./images/}}


\begin{document}

\begin{titlepage}
	\begin{center}
		\rule{\textwidth}{0.4mm}\\[0.5cm]
		\huge{\textbf{\TITLE}}
		\rule{\textwidth}{0.4mm}\\[0.5cm]
		\large{\THESISFORM}\\[0.5cm]
		\begin{minipage}[t]{0.4\textwidth}
			\begin{flushleft}
				\large\emph{Author}\\{\AUTHOR}
			\end{flushleft}
		\end{minipage}
		\begin{minipage}[t]{0.4\textwidth}
			\begin{flushright}
				\large\emph{Supervisor}\\{\SUPERVISOR}\\~\\
				\large\emph{Second Assessor}\\{\SECONDASSESSOR}
			\end{flushright}
		\end{minipage}
		\vfill
		\large{\INSTITUTE}\\
		\large{\FACULTY}\\
		\large{\UNIVERSITY}\\~\\
		\includegraphics[width=1.5cm]{\UNIVERSITYLOGO}
	\end{center}
\end{titlepage}

\thispagestyle{plain}
\section*{Abstract}


\newpage
\thispagestyle{plain}
\section*{Populaire Samenvatting}


\tableofcontents


\chapter{Theory}

\section{The Two-Dimensional Ising Model}

The two-dimensional Ising model in zero-field was first solved exactly in 1944 by Lars Onsager.\cite{onsager:1944}
It describes a square lattice with nearest neighbour interactions, where each lattice point has with it associated a number (which we'll refer to as spin) which may either be +1 or -1 and was originally meant as a model for magnets.
The Hamiltonian is\cite{mccoy:1973}
\begin{align}
	H = -J_{1} \sum_{j=1}^{\mathcal{M}} \sum_{k=1}^{\mathcal{N}} \sigma_{j,k} \sigma_{j,k+1} - J_{2} \sum_{j=1}^{\mathcal{M}} \sum_{k=1}^{\mathcal{N}} \sigma_{j,k} \sigma_{j+1,k},
\end{align}
with \(\mathcal{M}\) and \(\mathcal{N}\) the extent of the lattice in the \(x\)- and \(y\)-directions respectively and \(J_{1}\) and \(J_{2}\) the interaction strength between neighbours in respectively the \(x\)- and \(y\)-directions.
In the case where the interaction strength in both directions is the same the Hamiltonian becomes\cite{newman:1999}
\begin{align}
	H = -J \sum_{\langle ij \rangle} \sigma_{i} \sigma_{j},
\end{align}
where the bracket denotes summation over nearest neighbours.\footnote{Note that naively applying this Hamiltonian to calculate the lattice energy overcounts the energy by a factor of 2 since each bond is counted twice.}
In the ferromagnetic ground state (\(J>0\)) all spins on the lattice are aligned in one of two possible directions (the direction is chosen when the mirror symmetry in the lattice plane is spontaneously broken). 
The Hamiltonian is subject to toroidal boundary conditions in both directions, meaning \(\sigma_{1,k} = \sigma_{\mathcal{M}+1,k}\) and \(\sigma_{j,1} = \sigma_{j,\mathcal{N}+1}\).
We are interested in the thermodynamic properties of the Ising Model.
To that end we define the partition function
\begin{align}
	Z &= \sum_{\sigma = \pm 1} e^{-\beta H} \\
	  &= \sum_{\sigma = \pm 1} \prod_{j=1}^{\mathcal{M}} \prod_{k=1}^{\mathcal{N}} e^{\beta J_1 \sigma_{j,k} \sigma_{j,k+1}} \prod_{j=1}^{\mathcal{M}} \prod_{k=1}^{\mathcal{N}} e^{\beta J_2 \sigma_{j,k} \sigma_{j+1,k}},
\end{align}
with \(\beta=1/T\) (throughout this work we set the Boltzmann constant \(k_B\) equal to 1) and the sum running over every possible orientation of the spins on the lattice. To get an exact solution for this partition function some work is required.


% Set style to abbrv to get Vancouver style citations.
\bibliographystyle{abbrv}
\bibliography{bachelor-thesis}

\end{document}
